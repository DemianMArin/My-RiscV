Several optimizations and features can be added to improve the performance of the five-stage pipelined processor:

\begin{itemize}
    \item \textbf{Branch Prediction:} To reduce the penalty of control hazards, a branch predictor can be implemented. Instead of stalling the pipeline every time a branch is encountered, the processor can predict the outcome of the branch and speculatively fetch and execute instructions from the predicted path. If the prediction is correct, there is no penalty. If it is incorrect, the pipeline is flushed, but this is often better than always stalling.
    \item \textbf{Superscalar Execution:} A superscalar processor can execute more than one instruction per clock cycle by having multiple pipelines. For example, a 2-way superscalar processor could fetch, decode, and execute two instructions at a time, potentially doubling the IPC.
    \item \textbf{Out-of-Order Execution:} This allows the processor to execute instructions as soon as their operands are available, rather than strictly in program order. This can hide the latency of instructions with long execution times (like loads from memory) and improve the utilization of the execution units.
    \item \textbf{More Advanced Forwarding:} While the current implementation has forwarding, more complex forwarding paths can be added to handle more types of data hazards without stalling.
    \item \textbf{Cache Hierarchy:} Adding a multi-level cache (L1, L2, L3) can significantly reduce the average memory access time. A faster cache can provide data to the processor much more quickly than main memory, reducing the number of stalls due to memory access.
\end{itemize}
