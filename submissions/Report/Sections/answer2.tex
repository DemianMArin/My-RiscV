% Five-Stage Pipelined Processor Diagram
%
% This diagram shows a five-stage pipelined RISC-V processor.
% Pipelining increases throughput by overlapping the execution of multiple instructions.
%
% Stages:
% - IF (Instruction Fetch): Fetches the instruction from memory.
% - ID (Instruction Decode): Decodes the instruction and reads from the register file.
% - EX (Execute): Performs the ALU operation.
% - MEM (Memory Access): Reads from or writes to data memory.
% - WB (Write-Back): Writes the result back to the register file.
%
% Pipeline Registers (IF/ID, ID/EX, EX/MEM, MEM/WB):
% - These registers are placed between stages to hold the intermediate results and control signals for the instruction being processed in that stage.
%
% Hazard Unit:
% - Detects data and control hazards. It can stall the pipeline or control the forwarding paths to resolve these hazards.
%
% Forwarding Paths:
% - These paths (dashed lines) allow the result of an instruction in a later stage (EX or MEM) to be used as an operand for an instruction in an earlier stage (ID/EX), avoiding unnecessary stalls.

\begin{figure}[H]
    \centering
    \resizebox{\textwidth}{!}{
    \begin{tikzpicture}[
        node distance=1cm,
        auto,
        >=stealth',
        block/.style={draw, rectangle, minimum width=2cm, minimum height=1cm},
        pipeline/.style={draw, rectangle, minimum height=2.5cm}
    ]
        % Nodes
        \node (if) [block] {IF};
        \node (if_id) [pipeline, right=0.5cm of if] {IF/ID};
        \node (id) [block, right=0.5cm of if_id] {ID};
        \node (id_ex) [pipeline, right=0.5cm of id] {ID/EX};
        \node (ex) [block, right=0.5cm of id_ex] {EX};
        \node (ex_mem) [pipeline, right=0.5cm of ex] {EX/MEM};
        \node (mem) [block, right=0.5cm of ex_mem] {MEM};
        \node (mem_wb) [pipeline, right=0.5cm of mem] {MEM/WB};
        \node (wb) [block, right=0.5cm of mem_wb] {WB};

        % Connections
        \draw[->] (if) -- (if_id);
        \draw[->] (if_id) -- (id);
        \draw[->] (id) -- (id_ex);
        \draw[->] (id_ex) -- (ex);
        \draw[->] (ex) -- (ex_mem);
        \draw[->] (ex_mem) -- (mem);
        \draw[->] (mem) -- (mem_wb);
        \draw[->] (mem_wb) -- (wb);

        % Hazard Unit
        \node (hazard) [draw, rectangle, below=2cm of id] {Hazard Unit};
        \draw[->, dotted] (id.south) -- (hazard);
        \draw[->, dotted] (hazard.west) -| (if_id.south);
        \draw[->, dotted] (hazard.east) -| (id_ex.south);

        % Forwarding paths
        \draw[->, dashed] (ex_mem.south) -- ++(0,-0.75) -| (ex.south) node[midway, right, yshift=-0.2cm] {Forward};
        \draw[->, dashed] (mem_wb.south) -- ++(0,-1.25) -| (ex.south) node[midway, right, yshift=-0.2cm] {Forward};

    \end{tikzpicture}
    }
    \caption{Five Stage Pipelined Processor}
    \label{fig:five_stage}
\end{figure}
